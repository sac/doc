%% This document explains device files on Solaris/illumos
\documentclass[11pt]{article}
\title{Device files on illumos}
\author{Sachidananda Urs}
\date{}
\begin{document}
\maketitle

\begin{abstract}
  This document briefly explains the concept of device files and their
  management in the illumos operating system.
\end{abstract}

\subsection*{Introduction}

Commands that are used to manage the devices on an illumos machine expect device
name as a parameter. You need to know how to specify device names when using
commands to manage disks, file systems, and other devices. In most cases, you
can use logical device names to represent devices that are connected to the
system. Both logical and physical device names are represented on the system by
{\it logical} and {\it physical} device files.

\subsection*{Device information creation}

When a system is booted for the first time, a device hierarchy is created to
represent all the devices connected to the system. The kernel uses the device
hierarchy information to associate drivers with their appropriate devices. The
kernel also provides a set of pointers to the drivers that perform specific
operations.

Devices are represented in the file system by special files. In illumos, these
files reside in the /devices directory hierarchy.

Special files can be of type {\bf block} or {\bf character}. The type indicates
which kind of device driver operates the device. Drivers can be implemented to
operate on both types. For example, disk drivers export a character interface
for use by the fsck(1) and mkfs(1) utilities, and a block interface for use by
the file system.

Associated with each special file is a device number (dev\_t). A device number
consists of a {\bf major number} and a {\bf minor number}. The major number
identifies the device driver associated with the special file. The minor number
is created and used by the device driver to further identify the special
file. Usually, the minor number is an encoding that is used to identify which
device instance the driver should access and which type of access should be
performed. For example, the minor number can identify a tape device used for
backup and can specify that the tape needs to be rewound when the backup
operation is complete.

\subsection*{Device management}

The {\it devfs} file system manages the /devices directory, which is the name
space of all devices on the system. This directory represents the {\bf physical}
devices that consists of actual bus and device addresses.

The dev file system manages the /dev directory, which is the name space of
logical device names.

By default, the {\it devfsadm} command attempts to load every driver in the
system and attach to all possible device instances. Then, devfsadm creates the
device files in the /devices directory and the logical links in the /dev
directory. The devfsadm command also maintains the path\_to\_inst instance
database.

Updates to the /dev and /devices directories in response to dynamic
reconfiguration events or file system accesses are handled by devfsadmd, the
daemon version of the devfsadm command. This daemon is started by the service
management facility when a system is booted.

Because the devfsadmd daemon automatically detects device configuration changes
generated by any reconfiguration event, there is no need to run this command
interactively.

For more information, see the following references:

\begin{itemize}
\item devfsadm(1M)
\item dev(7FS)
\item devfs(7FS)
\item path\_to\_inst(4)
\end{itemize}

{\bf Device-to-driver mapping} --- The kernel maintains the device tree. Each
node in the tree represents a virtual or a physical device. The kernel binds
each node to a driver by matching the device node name with the set of drivers
installed in the system. The device is made accessible to applications only if
there is a driver binding.

\subsection*{Device naming convention}

Devices are referenced in three ways

\begin{description}
\item [Physical device name] -- Represents the full device path name in the
  device information hierarchy. The physical device name is created by when the
  device is first added to the system. Physical device files are found in the
  /devices directory.

\item [Instance name] -- Represents the kernel's abbreviation name for every
  possible device on the system. For example, sd0 and sd1 represent the instance
  names of two disk devices. Instance names are mapped in the
  /etc/path\_to\_inst file.

\item [Logical device name] -- The logical device name is created by when the
  device is first added to the system. Logical device names are used with most
  file system commands to refer to devices. Logical device files in the /dev
  directory are symbolically linked to physical device files in the /devices
  directory.
\end{description}

The preceding device name information is displayed with the following commands:
\begin{itemize}
\item dmesg
\item format
\item sysdef
\item prtconf
\end{itemize}

\subsection*{Overview of the Device Tree}

Devices in illumos are represented as a tree of interconnected device
information nodes. The device tree describes the configuration of loaded devices
for a particular machine.

The system builds a tree structure that contains information about the devices
connected to the machine at boot time. The device tree can also be modified by
dynamic reconfiguration operations while the system is in normal operation. The
tree begins at the root device node, which represents the platform.

Below the root node are the branches of the device tree. A branch consists of
one or more {\it bus nexus devices} and a terminating leaf device.

A bus nexus device provides bus mapping and translation services to subordinate
devices in the device tree. PCI - PCI bridges, PCMCIA adapters, and SCSI HBAs
are all examples of {\it nexus devices}. The discussion of writing drivers for
nexus devices is limited to the development of SCSI HBA drivers (see Chapter 18,
SCSI Host Bus Adapter Drivers).

Leaf devices are typically peripheral devices such as disks, tapes, network
adapters, frame buffers, and so forth. Leaf device drivers export the
traditional character driver interfaces and block driver interfaces. The
interfaces enable user processes to read data from and write data to either
storage or communication devices.

The system goes through the following steps to build the tree:

\begin{enumerate}
\item The CPU is initialized and searches for firmware.
\item The main firmware (OpenBoot, Basic Input/Output System (BIOS), or
  Bootconf) initializes and creates the device tree with known or
  self-identifying hardware.
\item When the main firmware finds compatible firmware on a device, the main
  firmware initializes the device and retrieves the device's properties.
\item The firmware locates and boots the operating system.
\item The kernel starts at the root node of the tree, searches for a matching
  device driver, and binds that driver to the device.
\item If the device is a nexus, the kernel looks for child devices that have
  not been detected by the firmware. The kernel adds any child devices to the
  tree below the nexus node.
\end{enumerate}

The kernel repeats the process from Step 5 until no further device nodes need to
be created.

Each driver exports a device operations structure dev\_ops(9S) to define the
operations that the device driver can perform. The device operations structure
contains function pointers for generic operations such as attach(9E),
detach(9E), and getinfo(9E). The structure also contains a pointer to a set of
operations specific to bus nexus drivers and a pointer to a set of operations
specific to leaf drivers.

The tree structure creates a parent-child relationship between nodes. This
parent-child relationship is the key to architectural independence. When a leaf
or bus nexus driver requires a service that is architecturally dependent in
nature, that driver requests its parent to provide the service. This approach
enables drivers to function regardless of the architecture of the machine or the
processor.

\subsubsection*{Displaying the Device Tree}

The device tree can be displayed in three ways:

\begin{itemize}

\item The libdevinfo library provides interfaces to access the contents of the
  device tree programmatically.
\item The prtconf(1M) command displays the complete contents of the device
  tree.
\item The /devices hierarchy is a representation of the device tree. Use the
  ls(1) command to view the hierarchy.
\end{itemize}

{\bf Note:}
/devices displays only devices that have drivers configured into the system. The
prtconf(1M) command shows all device nodes regardless of whether a driver for
the device exists on the system.

{\bf libdevinfo Library}
The libdevinfo library provides interfaces for accessing all public device
configuration data. See the libdevinfo(3LIB) man page for a list of interfaces.

\subsection*{References}
\begin{itemize}
\item[] http://illumos.org/books/wdd/kernelovr-77198.html
\item[] http://docs.oracle.com/cd/E23824\_01/html/821-1459/devaccess-90138.html
\end{itemize}

\end{document}
