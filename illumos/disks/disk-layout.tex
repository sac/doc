\documentclass{article}
\title{Solaris disk naming conventions\footnote{http://multiboot.solaris-x86.org/iv/3.html}}
\author{}
\date{}
\begin{document}
\maketitle

\begin{abstract}
  This document briefly explains how Solaris/illumos name the disk
  partitions. And a comparison is made with GNU/Linux systems. Note that this is
  legacy, newer operating system versions use ZFS and it is all new way of doing
  things which will be discussed in another document.
\end{abstract}

\section*{Disk addressing conventions}

On GNU/Linux systems SATA disks are named sda, sdb, sdc ... so on. And
the partitions on the disk are numbered from 1, i.e sda1, sda2, sda3 ... so
on.\vspace{0.1in}

On Solaris/illumos systems the disks are named using a different
convention. Combination of Disk {\bf c}ontroller, SCSI {\bf t}arget, {\bf d}isk
number, {\bf s}lice number, and {\bf p}artition number is used. Syntax for disk
entries on Solaris is shown below:\vspace{0.1in}

{\bf c}C [ {\bf t}T ] {\bf d}D \( {\bf s}S \ | \ {\bf p}P \) \vspace{0.1in}
\\
If you note above {\em t} appears only if you have a {\em SCSI} disk. \vspace{0.1in}

Where {\em C, T, D, S,} and {\em P} are non-negative decimal numbers. Actually,
a disk entry is a complete address of a slice or a partition, and consists of
two parts: logical device address formed by {\bf c, t, d,} and slice {\bf s} or
partition {\bf p} address on that device.\vspace{0.1in}

For example if you have a single SATA disk, your partitions might look like:\vspace{0.1in}

{\bf /dev/rdsk/c1d0s1 ... /dev/rdsk/c1d0s16} or

{\bf /dev/rdsk/c1d0p1 ... /dev/rdsk/c1d0p4}

\begin{description}

\item[C] is the logical controller number, i.e. the number of an IDE adapter
(actually its channel), SCSI host bus adapter or the like. It is assigned by
disks utility, or by devfsadm on newer systems. First time the utility is run
during Solaris installation, and consecutive numbers are assigned for particular
disk controllers, so that IDE controllers take precedence before SCSI ones. The
utility may be invoked later during system reconfiguration or manually by the
superuser to add new disks. Then, consecutive numbers are assigned to the new
controllers found on the system. E.g. if Solaris is installed on a machine with
only primary IDE channel being in use and one SCSI adapter they have numbers 0
and 1 respectively, when a new IDE device is attached to a free secondary
channel it will get the number 2.

\item[T] is the target (logical unit number) of the SCSI device, i.e. the
identification number of the device in a SCSI chain. Obviously, IDE disks have
no t specified in their addresses.

\item[D] is the number of the disk attached to the controller. Note that in case of an
IDE adapter, two IDE devices may be attached to a single channel, then 0
represents the master device, and 1 the slave one.

\item[S] is the number of the slice within a Solaris partition. Each Solaris partition
can have up to 16 slices numbered 0 through 15. There are 21 special files per a
disk device, 0 through 15 correspond to slices within the first Solaris
partition on a disk (see Slices and VTOC).

\item[P] equal to 0 means entire disk, values 1, 2, 3, 4 represent consecutive
partitions on a disk. These 5 entries correspond to disk special files 16
through 20. Pp is the FDISK partition number used by fdisk(1M).

\end{description}

\section*{Slices and VTOC}

Every Solaris partition is divided into smaller chunks called slices. They
resemble x86 fdisk partitions of hard disk drive. Up to 16 slices may be defined
per a Solaris partition. Only slices within the first Solaris partition can be
addresses by s0 through s15. Note however that there are still up to 4 Solaris
partitions per a disk allowed.

Some slices are reserved or have special meaning. Slice 2 is a backup slice
representing entire Solaris partition. Slice 8 is a boot slice representing the
first cylinder of the Solaris partition, containing partition boot block pboot,
Solaris disk label, VTOC and boot manager program bootblk. Slices 2 and 8 are
unmountable. Slice 9 is an alternates slice that uses 2 cylinders. Slices 2, 8
and 9 are reserved and should not be redefined. Slice 0 is usually a root slice
containing the root file system. The following tags may be assigned to slices:
swap, usr, stand, var and home (see fmthard man page).

There are 4 possible states of a slice. They are determined by two flags
specified when the slice is defined. One determines if the slice is mountable,
the other if it is read-only.

Slices can be defined using fmthard or interactive format utility. format always
shows 0 through 9 slices, including undefined ones. To define a slice we need to
specify its number, tag, flag, starting sector relative to Solaris partition and
its size.

All the information about slices is kept in a Volume Table Of Contents, shortly
VTOC. Together with Solaris disk label VTOC occupies the second and third sector
of a boot slice - the first cylinder of a Solaris partition. It can be printed
out with prtvtoc utility. It is recommended to print and save all
VTOCs\footnote{http://illumos.org/man/1M/prtvtoc - for printing VTOC} after
the system is setup to have them handy in the case of emergency.

\section*{Exercise}
On your OpenIndiana machine run the command fdisk /dev/rdsk/c1d0p0 or maybe
/dev/rdsk/c0d0p0 or maybe something else... try figuring out. Do not write
anything to disk, just play around.

\end{document}
